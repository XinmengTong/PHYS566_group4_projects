%%% Group 4: Tahoe, Ksenia, Xinmeng
%%% Computational Physics
%%% Spring, 2017

%%%%%%%%%%%%%%%%%%%%%%%%%%%%%%%%%%%%%%%%%%%%%%%%%%%%%%%%%%%%%%%%%%%%%%%%%%%%%%%%
%%% This code will develop a LaTeX'd writeup of the first group project
%%% assignment for PHYS566.
%%%%%%%%%%%%%%%%%%%%%%%%%%%%%%%%%%%%%%%%%%%%%%%%%%%%%%%%%%%%%%%%%%%%%%%%%%%%%%%%

% PACKAGES AND OTHER DOCUMENT CONFIGURATIONS
\documentclass[10pt]{article}
\usepackage[english]{babel}
\usepackage[utf8]{inputenc}
\usepackage{amsmath,amsfonts,amssymb}
\usepackage{graphicx,xcolor}
\usepackage{subfig}
\usepackage{booktabs}
\usepackage[left=2cm,%
right=2cm,%
top=2cm,%
bottom=2cm,%
headheight=11pt,%
letterpaper]{geometry}%
\usepackage{fancyhdr}
\pagestyle{fancy}
\lhead{\small\sffamily\bfseries\leftmark}%
\chead{}%
\rhead{\small\sffamily\bfseries\rightmark}
\renewcommand{\headrulewidth}{1pt}
\renewcommand{\footrulewidth}{1pt}
%\graphicspath{{}}

% Article Information
\title{Random Walks, the Diffusion Equation, and Cluster Growth}
\author{Tahoe Schrader, Ksenia , Xinmeng \\PHYS566}
\date{}

% Begin writing document
\begin{document}
\maketitle

% Start with the abstract
\abstract{In this assignment we blah blah blah}

\section{Theory}
\label{sec:theory}

\subsection{Diffusion Equations and the Finite Difference Form}
\label{sec:diffusionequation}
The diffusion equation in $1D$ is written,
\begin{equation}
  \label{eq:diffusioneqn}
  \frac{\partial\rho(x,t)}{\partial t} = D\nabla^2\rho(x,t),
\end{equation}
where $D$ is the diffusion constant. Equation~\ref{eq:diffusioneqn} is turned into an iterable form by noting: $\rho(x,t) = \rho(i\Delta x, n\Delta t) = \rho(i,n)$. This is the finite difference form\footnote{The finite difference form must be used because the diffusion equation is time dependent. Therefore, a relaxation method cannot be used.}.

After using the formal definition of derivatives and algebraically manipulating Equation~\ref{eq:diffusioneqn} in the finite difference form, we get
\begin{equation}
  \label{eq:diffusioneqn-iterable}
  \rho(i,n+1) = \rho(i,n) + \frac{D\Delta t}{\Delta x^2}\left(\rho(i+1,n) + \rho(i-1,n) - 2\rho(i,n)\right),
\end{equation}
where $\Delta t$ and $\Delta x$ are the step sizes in an iteration. This solution requires knowledge of initial conditions. We must assume that the $x$ displacement is known at times prior to and including $t_n = n\Delta t$. Two consecutive steps prior to the first unknown step is sufficient to solve such an equation.

We will use an initial density profile that is sharply peaked around $x=0$, but extends over a few grid sites to resemble a box. This is sufficient for generating the solution to the diffusion equation. Interestingly, after a couple iterations, the box profile will diffuse into a Gaussian normal distribution. The $1D$ Gaussian normal distribution has the form,
\begin{equation}
  \label{eq:gaussiandistribution}
  \rho(x,t) = \frac{1}{\sqrt{2\pi\sigma(t)^2}}\exp\left(-\frac{x^2}{2\sigma(t)^2}\right),
\end{equation}
where $\sigma(t) = \sqrt{2Dt}$.

The spatial expectation value, $\langle x(t)^2\rangle$, of Equation~\ref{eq:gaussiandistribution} is equal to $\sigma(t)^2$.
%%% NEED TO SHOW THIS IS TRUE. WILL PUT THAT PROOF HERE

\section{Computations}
\label{sec:computations}

\subsection{Diffusion of a Box Density Distribution}
\label{sec:diffusion-boxdensity}
Using the arguments in Section~\ref{sec:diffusionequation}, we solve the $1D$ Diffusion Equation over a period of time. Five different snapshots in time were then fit against Equation~\ref{eqn:gaussiandistribution} to show a box shaped density will eventually diffuse into a Gaussian normal distribution.

%%% PUT THAT PLOT HERE

\end{document} % This ends our document

% INCOMPLETE
